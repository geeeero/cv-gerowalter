\documentclass[a4paper]{simplecv}

\usepackage[utf8]{inputenc}
\usepackage[OT1]{fontenc}
\usepackage{graphicx}
\usepackage[ngerman]{babel}
\usepackage{url}
\usepackage{hyperref}
\usepackage{color}

\usepackage{natbib}
%\bibpunct{(}{)}{;}{a}{,}{,}
%\usepackage[fixlanguage]{babelbib}
%\selectbiblanguage{english}
\usepackage[final]{pdfpages}

\parindent0pt

\textwidth14.5cm
\oddsidemargin1.3cm

\def\FormatName#1{%
  \def\myname{Gero Walter}%
  \edef\name{#1}%
  \ifx\name\myname
    \textbf{#1}%
  \else
    #1%
  \fi
}



\begin{document}

\pagestyle{myheadings}
%\markright{Gero Walter, Department of Industrial Engineering \& Innovation Sciences, TU Eindhoven \hfill}
\markright{Dr.\ rer.\ nat.\ Gero Walter, Diplom-Statistiker \hfill}


%\leftheader{
%Institut f?r Statistik\\
%Ludwig-Maximilians-Universit?t M?nchen\\
%Tel: +49 89 2180 3698
%}

%\rightheader{
%\url{Gero.Walter@stat.uni-muenchen.de}\\
%\url{www.stat.uni-muenchen.de/~walter}
%\texttt{\small Gero.Walter@stat.uni-muenchen.de}\\
%\texttt{\small www.stat.uni-muenchen.de/\,$\tilde{}\;$walter}
%}

\title{Lebenslauf}


\vspace*{-10ex}
\maketitle

\textbf{\large Dr.\ rer.\ nat.\ Gero Walter, Diplom-Statistiker}\\[2ex]
\url{http://www.geeeero.de}\\
\url{mailto:g.m.walter@tue.nl} \hfill Tel: +49 174 639 56 13\\
%url{mailto:gero@gmx.at}       \hfill Tel: +31 638 377 678\\

Geboren am 26.06.1979 in T\"{u}bingen.
%Parents: Bernhard Walter, Christa Walter nee Wend.\\%[1ex]
%Sister Simone, born 1976; brother Jan, born 1981.\\%[1ex]
%Marital status: unmarried.

%\vspace*{-2ex}
\section{Überblick}
\begin{topic}
\item[\bfseries 2015] Postdoctoral Researcher an der TU Eindhoven (NL):\\
Eigenständige Forschung und Betreuung von Abschlussarbeiten zu\\ Reliability sowie Planung und Optimierung von Wartungsarbeiten.
\item[\bfseries 2014] Temporary Lecturer an der Universität Durham (UK):\\
Dozent f\"{u}r Vorlesungen zu Bayes-Statistik und Entscheidungstheorie.
\item[\bfseries 2013] Dr.\ rer.\ nat.\ in Statistik an der Ludwig-Maximilians-Universität München:
Dissertation zu Bayesianische Methoden mit Mengen von Priori-Verteilungen.
\item[\bfseries 2007] Diplom-Statistiker an der Ludwig-Maximilians-Universität München:
Diplomarbeit über Bayesianische Lineare Regression bei begrenztem Vorwissen.
\end{topic}


\section{Auszeichnungen}
\label{awards}
\begin{itemize}
\item Best poster award (1st Prize) bei der \href{http://www.sipta.org/isipta15/?pag=presentation}{ISIPTA'15} (mit F.~Coolen und S.D.~Flapper)
\item Best poster award (2nd Prize) bei der \href{http://www.sipta.org/isipta15/?pag=presentation}{ISIPTA'15} (mit C.~Jansen und T.~Augustin)
\item \href{http://www.sipta.org/isipta11/index.php?id=prizes_ijar}{IJAR Young Researcher Award 2011} (Silver Award)
\item Best poster award (1st Prize) bei der \href{http://www.geeeero.de/files/talk-2011b-ISIPTA-sna-bestposter-small.jpg}{ISIPTA'11} (mit M.~Eugster und T.~Augustin)
\end{itemize}
%\begin{topic}
%\item[7 / 2015] Best poster award (1st Prize) bei der \href{http://www.sipta.org/isipta15/?pag=presentation}{ISIPTA'15}\\
%                 (mit F.~Coolen und S.D.~Flapper);\\
%                 (Zusammenarbeit mit Frank Coolen und Simme Douwe Flapper);\\
%                Best poster award (2nd Prize) bei der \href{http://www.sipta.org/isipta15/?pag=presentation}{ISIPTA'15}\\
%                 (Zusammenarbeit mit Christoph Jansen und Thomas Augustin)
%\item[7 / 2011] \href{http://www.sipta.org/isipta11/index.php?id=prizes_ijar}{IJAR Young Researcher Award 2011} (Silver Award);\\
%                Best poster award (1st Prize) bei der \href{http://www.geeeero.de/files/talk-2011b-ISIPTA-sna-bestposter-small.jpg}{ISIPTA'11}\\
%                 (Zusammenarbeit mit Manuel Eugster und Thomas Augustin)
%\end{topic}

\section{Software} %kenntnisse
\label{software}
%Programmierung in \textsf{R}, SPSS, SAS, GNU/Linux, \texttt{git}, SQL, MS Office %(Versionskontrollsystem)
\begin{tabular}{ll}
{\sf \emph{Expertenkenntnisse}} &
\textsf{R} \\
{\sf \emph{Fortgeschrittene Kentnisse}} &
SPSS, SAS, GNU/Linux, \texttt{git} \\
{\sf \emph{Grundkenntnisse}} &
SQL, MS Office, Libre Office
\end{tabular}

%\vspace*{-2.5ex}
%\subsection{Languages}
\section{Sprachen}
\label{languages}
%\vspace*{-1.3ex}
%\vspace*{-1ex}
\begin{tabular}{llllll}%
{\sf \emph{Deutsch}}    & Muttersprache \qquad\qquad\qquad\qquad &
{\sf \emph{Isländisch}} & fortgeschrittene Kenntnisse \\
%{\sf \emph{Italienisch}}   & fließend \\
{\sf \emph{Englisch}}   & verhandlungsssicher &
{\sf \emph{Französisch}}    & fortgeschrittene Kenntnisse & \\
{\sf \emph{Italienisch}}   & fließend
\end{tabular}
%
\iffalse
\begin{topic}
\item[German] mother tongue
\vspace*{-1ex}
\item[English] fluently
\vspace*{-1ex}
\item[Italian] fluently
\vspace*{-1ex}
\item[French] intermediate
\vspace*{-1ex}
\item[Icelandic] basic
\end{topic}
\fi

\section{Weitere Informationen}
%\textbf{\sf Student project supervision} \dotfill \pageref{supervision}\\
%\textbf{\sf Teaching} \dotfill \pageref{teaching}\\
%\textbf{\sf Talks and Posters} \dotfill \pageref{talks}\\
%\textbf{\sf Other scientific activities} \dotfill \pageref{other}\\
\textbf{\sf Vorträge \& Poster:} \url{http://www.geeeero.de/forschung.html#talks}\\
\textbf{\sf Veröffentlichungen:} \url{http://www.geeeero.de/forschung.html#publications}\\
\textbf{\sf Ausbildung und Berufserfahrung}: \url{http://www.geeeero.de/lebenslauf.html} {\sf und nächste Seite}\\
%\phantom{a} \hfill {\sf und nächste Seite}\\

\newpage
\section{Ausbildung und Berufserfahrung}
\label{education}

\begin{topic}
\item[\hspace*{-2ex}\bfseries 2015 -- 2016] Postdoctoral Researcher
                  in der Arbeitsgruppe \href{http://opac.ieis.tue.nl/}{\glqq Operations, Planning, Accounting, and Control\grqq} (OPAC)
                  am \href{http://www.tue.nl/en/university/departments/industrial-engineering-innovation-sciences/}{Department of Industrial Engineering \& Innovation Sciences},
                  \href{http://www.tue.nl/en/}{TU Eindhoven} (NL).
\begin{itemize}
\item Forschung im Rahmen des \href{http://www.dinalog.nl/en/}{TKI DINALOG}-Projekts
\href{http://www.dinalog.nl/en/project/campi/}{\glqq Coordinated Advanced Maintenance and Logistics Planning for the Process Industries\grqq} (CAMPI).
\item Betreuung von Abschlussarbeiten in Kooperation mit
\href{http://www.bp.com/nl_nl/netherlands/over-bp/onze-raffinaderij.html}{BP},
\href{http://www.daf.com/EN}{DAF Trucks},
\href{http://www.fei.com/locations/fei-netherlands/}{FEI},
\href{http://frankfurt-consulting.de}{Frankfurt Consulting Engineers},
\href{http://www.frieslandcampina.com/en/}{FrieslandCampina},
\href{http://marel.com/poultry-processing}{Marel Poultry Processing},
\href{http://www.nedtrain.nl/}{NedTrain},
\href{http://www.defensie.nl/english/organisation/navy}{Royal Netherlands Navy},
\href{http://www.sabic.com}{Sabic},
\href{http://www.sitech.nl/en}{Sitech},
\href{http://www.smurfitkappa.com/vHome/nl/Roermond}{Smurfit Kappa},
\href{http://www.stork.com/en}{Stork},
\href{http://www.tatasteel.nl}{Tata Steel},
\href{http://www.vdlindustrialmodules.nl}{VDL}.
\end{itemize}

\item[\hspace*{-2ex}\bfseries 1 -- 12 / 2014] Temporary Lecturer (Dozent)
                  am \href{http://www.dur.ac.uk/mathematical.sciences/}{Department of Mathematical Sciences},
                  \href{http://www.durham.ac.uk}{Universität Durham} (UK).
                  Vorlesungen zu Bayes-Statistik und Entscheidungstheorie,
                  Übungen zu Wahrscheinlichkeitstheorie und Computerübungen. 

\item[\hspace*{-2ex}\bfseries 10 / 2013] Dr.\ rer.\ nat. in Statistik
                  (\href{http://www.geeeero/files/drrernat-de.pdf}{magna cum laude}) an der
                  \href{http://www.lmu.de}{Lud\-wig-Maximi\-li\-ans-Uni\-ver\-si\-tät} (LMU) München,
                  mit der Dissertation \href{https://edoc.ub.uni-muenchen.de/17059/}{``Generalized Bayesian Inference under Prior-Data Conflict''}.
                  %(\cite{diss} in the List of publications, page~\pageref{publications}).


\item[\hspace*{-2ex}\bfseries 2007 -- 2013] Doktorand in der Arbeitsgruppe
                    \href{http://www.statistik.lmu.de/institut/ag/agmg/index.html}{\glqq Method(olog)ische Grundlagen der Statistik und ihre Anwendungen\grqq}
                    (\href{http://www.statistik.lmu.de/~thomas/}{Prof.\ Dr.\ Thomas Augustin}),
                    \href{http://www.statistik.lmu.de/index.html}{Institut für Statistik}, \href{http://www.lmu.de/}{LMU München}.
%                    Responsibilities included teaching for statistics students and other majors, %postgraduate ???
%                    assistance in supervision of Bachelor's theses,
%                    administrative work and student guidance related to the department's Bachelor's and Master's study programmes.
%                    Responsibilities assumed at the Department of Statistics:
\begin{itemize}
\item Übungen für Haupt- und Nebenfachstudierende.
\item Betreuung von Bachelorarbeiten.
\item Studienberatung und Administrationsaufgaben zu den Bachelor- und Masterstudiengängen am Institut für Statistik.
\end{itemize}

\item[2 -- 4 / 2010] Gastwissenschaftler am \href{http://www.dur.ac.uk/mathematical.sciences/}{Department of Mathematical Sciences},
                 \href{http://www.durham.ac.uk}{Universität Durham} (UK),
                 auf Einladung von \href{http://maths.dur.ac.uk/stats/people/fc/fc.html}{Prof.\ Dr.\ Frank Coolen} (zwei Monate).
                 2011, 2013 und 2016 weitere Forschungsaufenthalte in Durham.

%\item[7 / 2010] Participant of the \emph{4th SIPTA School on Imprecise Probabilities}, Durham. %, UK.

%\item[7 / 2008] Participant of the \emph{3rd SIPTA School on Imprecise Probabilities}, Montpellier. %, France.

\item[\hspace*{-2ex}\bfseries 5 / 2007] Diplom-Statistiker (\href{http://www.geeeero.de/files/diplom.pdf}{Gesamtnote 1.37}) mit der Diplomarbeit
\href{http://www.geeeero/files/Diplomarbeit_GeroWalter.pdf}{\glqq Bayes-\-Re\-gres\-sion mit Mengen von Prioris - Ein Beitrag zur Statistik unter komplexer Unsicherheit\grqq}\
(Note 1.0, Betreuung \href{http://www.statistik.lmu.de/~thomas/}{Prof.\ Dr.\ Thomas Augustin}).

\item[\hspace*{-2ex}\bfseries 2000 -- 2007] Studium im Di\-plom-Studium Statistik der LMU München,
mit den Anwendungsfächern Politikwissenschaften und Soziologie.

%\item[2005 -- 2007] Projektarbeit für die HelmholzZentrum-Forschungsgruppe \glqq Epidemiologie von Luftschadstoffen\grqq.

\item[7 -- 9 / 2005] Praktikum am \href{http://www.helmholtz-muenchen.de/}{HelmholzZentrum München} (ehemals GSF),
Institut für Epidemiologie, Forschungsgruppe \glqq Epidemiologie von Luftschadstoffen\grqq\
(Leitung \href{http://www.helmholtz-muenchen.de/epi2/the-institute/staff/staff/ma/362/Prof.%20Dr.-Peters/index.html}{Dr.\ Annette Peters}).
Im Anschluss bis 2007 weitere Projektarbeit für die Forschungsgruppe.

\item[2004 -- 2007] Wissenschaftliche Hilfskraft (HiWi) am
\href{http://www.statistik.lmu.de/index.html}{Institut für Statistik}, \href{http://www.lmu.de/}{LMU München},
bei \href{http://www.statistik.lmu.de/~thomas/}{Prof.\ Dr.\ Thomas Augustin}.

\item[2003] Wahl zum Studentenvertreter im Fakultätsrat der
\href{http://www.mathematik-informatik-statistik.uni-muenchen.de/}{Fakultät für Mathematik, Informatik und Statistik}; 2004 wiedergewählt.

\item[2002 -- 2003] Erasmus-Aufenthalt an der \href{http://www.unipa.it/}{Universität Palermo} (zwei Semester). %Two semester
%\item[2001 -- 2007] Mitglied der Fachschaft Statistik: Organisation von Einführungsveranstaltungen und der Bundesfachschaftentagung Statistik.
%\item[1990 -- 1999] Secondary School (Eugen-Bolz-Gymnasium) in Rottenburg am Neckar.
%\item[1986 -- 1990] Primary school (Hohenberg-Grundschule) in Rottenburg am Ne\-ckar.
\end{topic}

\includepdf[pagecommand={\thispagestyle{empty}},pages={1}]{drrernat-de.pdf}
\includepdf[pagecommand={\thispagestyle{empty}},pages={1}]{drrernat-de-zeugnis.pdf}
\includepdf[pagecommand={\thispagestyle{empty}},pages={1-3}]{diplom.pdf}

%
%\begin{picture}
%\put(0,0){ABC}
%\end{picture
%%%%%%%%%%%%%%%%%%%%%%%%%%%%%%%%%%
%\vspace*{-14.1cm}%
%{\color{white}\rule{12ex}{2.2ex}}
%%%%%%%%%%%%%%%%%%%%%%%%%%%%%%%%%%
%\rule{12ex}{2.2ex}

\end{document}
